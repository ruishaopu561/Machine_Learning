\documentclass{article} % For LaTeX2e
\usepackage{nips15submit_e,times}
\usepackage{hyperref}
\usepackage{url}
%\documentstyle[nips14submit_09,times,art10]{article} % For LaTeX 2.09


\title{Formatting Instructions for NIPS 2015}


% The \author macro works with any number of authors. There are two commands
% used to separate the names and addresses of multiple authors: \And and \AND.
%
% Using \And between authors leaves it to \LaTeX{} to determine where to break
% the lines. Using \AND forces a linebreak at that point. So, if \LaTeX{}
% puts 3 of 4 authors names on the first line, and the last on the second
% line, try using \AND instead of \And before the third author name.

\newcommand{\fix}{\marginpar{FIX}}
\newcommand{\new}{\marginpar{NEW}}

%\nipsfinalcopy % Uncomment for camera-ready version

\begin{document}


\maketitle

\begin{abstract}
This week I spend some time to build my environment on Linux.
\end{abstract}

\section{Works}
Since the main memory of my computer is only 4G, it's really slow to run code on Linux, which is installed on HDD. So I used to run code in Windows. However, it's troublesome to configure some environment variables. So i decide to run code on Linux now. This week I just spend some time to install all the corresponding softwares and configur correctly. I also try to build a sample network from Alex Net. Although it wworks with many bugs.
\begin{center}
   \url{http:
\end{center}



\subsection{NVIDIA} 
  用```lspci | grep -i nvidia``查看GPU版本信息
\subsection{Anaconda3} 
Download the Anaconda3 installation package which python version is 3.7 on the following website. Install it as requested.
\begin{center}
	\url{https://www.anaconda.com}
\end{center}
  7 因为我的terminal做了优化,用的是zsh,所以遇到```zsh: command not found: conda```的问题时,要在```~/.zshrc`
  8 ```
  9 export PATH="/home/rsp/anaconda3/bin:\$PATH
 10 ```
 11 然后
 12 ```
 13 source ~/.zshrc
 14 ```
 15 再次激活,就可以了。
\subsection{PyTorch}
\subsubsection{Missing dependencies for SOCKS support}
 18 在PyCharm安装python模块(不仅仅是安装模块)出现如下错误,终端使用pip3命令也是同样的错误。这是因为本地设置了s
 19 ```
 20 Could not install packages due to an EnvironmentError: Missing dependencies for SOCKS support.
 21 ```
\subsubsection{Sulotion}
 26 1.安装pysocks模块,配合socks代理工具使用。

 28 Unset socks proxy

 30 unset all_proxy
 31 unset ALL_PROXY
 
 33 Install missing dependencies
 
 35 pip install pysocks
 
 37 Reset proxy
 
 39 source ~/.bashrc
 
\subsection{Referenced website}
https://blog.csdn.net/dcrmg/article/details/78146797\\
https://blog.csdn.net/codechelle/article/details/77414117\\
https://blog.tearth.me/python_installmodule/

\subsection{virtualenv}
  7 ```
  8 pip install virtualenv
  9 ```
\subsubsection{virtualenv基本用法}
 11 然后,假定我们要开发一个新的项目,需要一套独立的Python运行环境,可以这么做:
 
\subsubsection{新建项目文件夹}
 
 13 ```
 14 mkdir myproject
 15 cd myproject/
 16 ```

\subsubsection{创建一个独立的Python运行环境,命名为venv}

 18 ```
 19 virtualenv --no-site-packages venv
 20 ```

 21 命令```virtualenv```就可以创建一个独立的Python运行环境

 23 参数```–no-site-packages```,已经安装到系统Python环境中的所有第三方包都不会复制过来,这样,我们就得到了一>
 24 参数 ```–system-site-packages```,虚拟环境会继承```/usr/lib/python2.7/site-packages```下的所有库。如果想依

\subsubsection{激活虚拟环境}
 27 新建的Python环境被放到当前目录下的venv目录。有了venv这个Python环境,可以用```source```进入该环境:
 
 29 \$ source venv/bin/activate
 30 (venv)\$

 32 注意到命令提示符变了,有个(venv)前缀,表示当前环境是一个名为```venv```的Python环境。
 33 下面正常安装各种第三方包,并运行python命令:
 
 35 (venv)\$ pip install jinja2

 37 在venv环境下,用pip安装的包都被安装到venv这个环境下,系统Python环境不受任何影响。也就是说,venv环境是专门

\subsubsection{shutdown virtual environment}
 40 退出当前的venv环境,使用deactivate命令:

 42 \$ deactivate

 44 此时就回到了正常的环境,现在pip或python均是在系统Python环境下执行。

\subsubsection{指定python版本}
 47 可以使用```-p PYTHON_EXE```选项在创建虚拟环境的时候指定python版本

 48 ```zsh
 49 Mac:myproject johnson\$ virtualenv -p /usr/bin/python2.7 venv
 50 ```
 
 51 到此已经可以解决python版本冲突问题和python库不同版本的问题
\subsubsection{Referenced website}
 https://blog.csdn.net/John_xyz/article/details/80665320


\end{document}
